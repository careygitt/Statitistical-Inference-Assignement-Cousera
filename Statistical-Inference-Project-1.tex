% Options for packages loaded elsewhere
\PassOptionsToPackage{unicode}{hyperref}
\PassOptionsToPackage{hyphens}{url}
%
\documentclass[
]{article}
\usepackage{amsmath,amssymb}
\usepackage{iftex}
\ifPDFTeX
  \usepackage[T1]{fontenc}
  \usepackage[utf8]{inputenc}
  \usepackage{textcomp} % provide euro and other symbols
\else % if luatex or xetex
  \usepackage{unicode-math} % this also loads fontspec
  \defaultfontfeatures{Scale=MatchLowercase}
  \defaultfontfeatures[\rmfamily]{Ligatures=TeX,Scale=1}
\fi
\usepackage{lmodern}
\ifPDFTeX\else
  % xetex/luatex font selection
\fi
% Use upquote if available, for straight quotes in verbatim environments
\IfFileExists{upquote.sty}{\usepackage{upquote}}{}
\IfFileExists{microtype.sty}{% use microtype if available
  \usepackage[]{microtype}
  \UseMicrotypeSet[protrusion]{basicmath} % disable protrusion for tt fonts
}{}
\makeatletter
\@ifundefined{KOMAClassName}{% if non-KOMA class
  \IfFileExists{parskip.sty}{%
    \usepackage{parskip}
  }{% else
    \setlength{\parindent}{0pt}
    \setlength{\parskip}{6pt plus 2pt minus 1pt}}
}{% if KOMA class
  \KOMAoptions{parskip=half}}
\makeatother
\usepackage{xcolor}
\usepackage[margin=1in]{geometry}
\usepackage{color}
\usepackage{fancyvrb}
\newcommand{\VerbBar}{|}
\newcommand{\VERB}{\Verb[commandchars=\\\{\}]}
\DefineVerbatimEnvironment{Highlighting}{Verbatim}{commandchars=\\\{\}}
% Add ',fontsize=\small' for more characters per line
\usepackage{framed}
\definecolor{shadecolor}{RGB}{248,248,248}
\newenvironment{Shaded}{\begin{snugshade}}{\end{snugshade}}
\newcommand{\AlertTok}[1]{\textcolor[rgb]{0.94,0.16,0.16}{#1}}
\newcommand{\AnnotationTok}[1]{\textcolor[rgb]{0.56,0.35,0.01}{\textbf{\textit{#1}}}}
\newcommand{\AttributeTok}[1]{\textcolor[rgb]{0.13,0.29,0.53}{#1}}
\newcommand{\BaseNTok}[1]{\textcolor[rgb]{0.00,0.00,0.81}{#1}}
\newcommand{\BuiltInTok}[1]{#1}
\newcommand{\CharTok}[1]{\textcolor[rgb]{0.31,0.60,0.02}{#1}}
\newcommand{\CommentTok}[1]{\textcolor[rgb]{0.56,0.35,0.01}{\textit{#1}}}
\newcommand{\CommentVarTok}[1]{\textcolor[rgb]{0.56,0.35,0.01}{\textbf{\textit{#1}}}}
\newcommand{\ConstantTok}[1]{\textcolor[rgb]{0.56,0.35,0.01}{#1}}
\newcommand{\ControlFlowTok}[1]{\textcolor[rgb]{0.13,0.29,0.53}{\textbf{#1}}}
\newcommand{\DataTypeTok}[1]{\textcolor[rgb]{0.13,0.29,0.53}{#1}}
\newcommand{\DecValTok}[1]{\textcolor[rgb]{0.00,0.00,0.81}{#1}}
\newcommand{\DocumentationTok}[1]{\textcolor[rgb]{0.56,0.35,0.01}{\textbf{\textit{#1}}}}
\newcommand{\ErrorTok}[1]{\textcolor[rgb]{0.64,0.00,0.00}{\textbf{#1}}}
\newcommand{\ExtensionTok}[1]{#1}
\newcommand{\FloatTok}[1]{\textcolor[rgb]{0.00,0.00,0.81}{#1}}
\newcommand{\FunctionTok}[1]{\textcolor[rgb]{0.13,0.29,0.53}{\textbf{#1}}}
\newcommand{\ImportTok}[1]{#1}
\newcommand{\InformationTok}[1]{\textcolor[rgb]{0.56,0.35,0.01}{\textbf{\textit{#1}}}}
\newcommand{\KeywordTok}[1]{\textcolor[rgb]{0.13,0.29,0.53}{\textbf{#1}}}
\newcommand{\NormalTok}[1]{#1}
\newcommand{\OperatorTok}[1]{\textcolor[rgb]{0.81,0.36,0.00}{\textbf{#1}}}
\newcommand{\OtherTok}[1]{\textcolor[rgb]{0.56,0.35,0.01}{#1}}
\newcommand{\PreprocessorTok}[1]{\textcolor[rgb]{0.56,0.35,0.01}{\textit{#1}}}
\newcommand{\RegionMarkerTok}[1]{#1}
\newcommand{\SpecialCharTok}[1]{\textcolor[rgb]{0.81,0.36,0.00}{\textbf{#1}}}
\newcommand{\SpecialStringTok}[1]{\textcolor[rgb]{0.31,0.60,0.02}{#1}}
\newcommand{\StringTok}[1]{\textcolor[rgb]{0.31,0.60,0.02}{#1}}
\newcommand{\VariableTok}[1]{\textcolor[rgb]{0.00,0.00,0.00}{#1}}
\newcommand{\VerbatimStringTok}[1]{\textcolor[rgb]{0.31,0.60,0.02}{#1}}
\newcommand{\WarningTok}[1]{\textcolor[rgb]{0.56,0.35,0.01}{\textbf{\textit{#1}}}}
\usepackage{graphicx}
\makeatletter
\def\maxwidth{\ifdim\Gin@nat@width>\linewidth\linewidth\else\Gin@nat@width\fi}
\def\maxheight{\ifdim\Gin@nat@height>\textheight\textheight\else\Gin@nat@height\fi}
\makeatother
% Scale images if necessary, so that they will not overflow the page
% margins by default, and it is still possible to overwrite the defaults
% using explicit options in \includegraphics[width, height, ...]{}
\setkeys{Gin}{width=\maxwidth,height=\maxheight,keepaspectratio}
% Set default figure placement to htbp
\makeatletter
\def\fps@figure{htbp}
\makeatother
\setlength{\emergencystretch}{3em} % prevent overfull lines
\providecommand{\tightlist}{%
  \setlength{\itemsep}{0pt}\setlength{\parskip}{0pt}}
\setcounter{secnumdepth}{-\maxdimen} % remove section numbering
\ifLuaTeX
  \usepackage{selnolig}  % disable illegal ligatures
\fi
\IfFileExists{bookmark.sty}{\usepackage{bookmark}}{\usepackage{hyperref}}
\IfFileExists{xurl.sty}{\usepackage{xurl}}{} % add URL line breaks if available
\urlstyle{same}
\hypersetup{
  pdftitle={Statistical Inference Assignment},
  pdfauthor={Carey Raychelle},
  hidelinks,
  pdfcreator={LaTeX via pandoc}}

\title{Statistical Inference Assignment}
\author{Carey Raychelle}
\date{2023-06-19}

\begin{document}
\maketitle

\#\#Assignment Description Investigating exponential distribution in R
and comparing with the Central Limit Theorem. Exponential distribution
in R can be simulated by rexp(n,lamda) where lamda is rate.Mean and
standard deviation of distribution is 1/lamda Lamda is set to be 0.2

\begin{Shaded}
\begin{Highlighting}[]
\CommentTok{\#install the packages needed}
\FunctionTok{library}\NormalTok{(knitr)}
\end{Highlighting}
\end{Shaded}

\begin{verbatim}
## Warning: package 'knitr' was built under R version 4.2.3
\end{verbatim}

\begin{Shaded}
\begin{Highlighting}[]
\FunctionTok{library}\NormalTok{(ggplot2)}
\end{Highlighting}
\end{Shaded}

\begin{verbatim}
## Warning: package 'ggplot2' was built under R version 4.2.2
\end{verbatim}

\begin{Shaded}
\begin{Highlighting}[]
\CommentTok{\#making the report reproducible}
\FunctionTok{set.seed}\NormalTok{(}\DecValTok{12345}\NormalTok{)}
\end{Highlighting}
\end{Shaded}

\#Simulation Exercise In this project you will investigate the
exponential distribution in R and compare it with the Central Limit
Theorem. The exponential distribution can be simulated in R with rexp(n,
lambda) where lambda is the rate parameter. The mean of exponential
distribution is 1/lambda and the standard deviation is also 1/lambda.
Set lambda = 0.2 for all of the simulations. You will investigate the
distribution of averages of 40 exponentials. Note that you will need to
do a thousand simulations.

\begin{Shaded}
\begin{Highlighting}[]
\CommentTok{\#setting variations for the simulation}
\NormalTok{n }\OtherTok{\textless{}{-}} \DecValTok{40}
\NormalTok{lamda }\OtherTok{\textless{}{-}} \FloatTok{0.2}
\NormalTok{sim }\OtherTok{\textless{}{-}} \DecValTok{1000}
\CommentTok{\#Create a matrix with 1000 rows and columns 40 of random simulations}
\NormalTok{matsim }\OtherTok{\textless{}{-}} \FunctionTok{matrix}\NormalTok{(}\FunctionTok{rexp}\NormalTok{(sim }\SpecialCharTok{*}\NormalTok{ n,}\AttributeTok{rate =}\NormalTok{ lamda),sim,n)}
\CommentTok{\#Calculate the means and plot}
\NormalTok{sim\_mean }\OtherTok{\textless{}{-}} \FunctionTok{rowMeans}\NormalTok{(matsim)}
\FunctionTok{hist}\NormalTok{(sim\_mean, }\AttributeTok{xlab=}\StringTok{"Mean of 40 exponentials"}\NormalTok{,}\AttributeTok{ylab=}\StringTok{"Frequency"}\NormalTok{,}\AttributeTok{main=}\StringTok{"Histogram of Simulation Mean"}\NormalTok{,}\AttributeTok{col=}\StringTok{"purple"}\NormalTok{)}
\end{Highlighting}
\end{Shaded}

\includegraphics{Statistical-Inference-Project-1_files/figure-latex/unnamed-chunk-2-1.pdf}
\#Sample Mean and Theoretical Mean

\begin{Shaded}
\begin{Highlighting}[]
\NormalTok{sample\_mean }\OtherTok{\textless{}{-}} \FunctionTok{mean}\NormalTok{(sim\_mean)}
\NormalTok{sample\_mean}
\end{Highlighting}
\end{Shaded}

\begin{verbatim}
## [1] 4.971972
\end{verbatim}

\begin{Shaded}
\begin{Highlighting}[]
\NormalTok{theoritical\_mean }\OtherTok{\textless{}{-}} \DecValTok{1}\SpecialCharTok{/}\NormalTok{lamda}
\NormalTok{theoritical\_mean}
\end{Highlighting}
\end{Shaded}

\begin{verbatim}
## [1] 5
\end{verbatim}

From the above the sample mean which is 4.97192 is very close to the
theoretical mean which is 5

\#Sample Variance and Theoretical Variance

\begin{Shaded}
\begin{Highlighting}[]
\NormalTok{sample\_var }\OtherTok{\textless{}{-}} \FunctionTok{var}\NormalTok{(sim\_mean)}
\NormalTok{sample\_var}
\end{Highlighting}
\end{Shaded}

\begin{verbatim}
## [1] 0.6157926
\end{verbatim}

\begin{Shaded}
\begin{Highlighting}[]
\NormalTok{theoretical\_var }\OtherTok{\textless{}{-}}\NormalTok{ (}\DecValTok{1}\SpecialCharTok{/}\NormalTok{lamda)}\SpecialCharTok{\^{}}\DecValTok{2}\SpecialCharTok{/}\NormalTok{n}
\NormalTok{theoretical\_var}
\end{Highlighting}
\end{Shaded}

\begin{verbatim}
## [1] 0.625
\end{verbatim}

The sample variance is also close to the theoretical variance

\#\#Approximate Normal Distribution Creating an approximate normal to
see how the sample differs from it

\begin{Shaded}
\begin{Highlighting}[]
\NormalTok{plotnorm }\OtherTok{\textless{}{-}} \FunctionTok{data.frame}\NormalTok{(sim\_mean)}
\NormalTok{g }\OtherTok{\textless{}{-}} \FunctionTok{ggplot}\NormalTok{(plotnorm,}\FunctionTok{aes}\NormalTok{(}\AttributeTok{x=}\NormalTok{sim\_mean))}
\NormalTok{g}\OtherTok{=}\NormalTok{g}\SpecialCharTok{+}\FunctionTok{geom\_histogram}\NormalTok{(}\FunctionTok{aes}\NormalTok{(}\AttributeTok{y=}\FunctionTok{after\_stat}\NormalTok{(density)),}\AttributeTok{colour=}\StringTok{"green"}\NormalTok{,}\AttributeTok{fill=}\StringTok{"orange"}\NormalTok{)}
\NormalTok{g}\OtherTok{=}\NormalTok{g}\SpecialCharTok{+}\FunctionTok{geom\_density}\NormalTok{(}\AttributeTok{colour=}\StringTok{"black"}\NormalTok{, }\AttributeTok{size=}\DecValTok{1}\NormalTok{)}
\end{Highlighting}
\end{Shaded}

\begin{verbatim}
## Warning: Using `size` aesthetic for lines was deprecated in ggplot2 3.4.0.
## i Please use `linewidth` instead.
## This warning is displayed once every 8 hours.
## Call `lifecycle::last_lifecycle_warnings()` to see where this warning was
## generated.
\end{verbatim}

\begin{Shaded}
\begin{Highlighting}[]
\NormalTok{g}\OtherTok{=}\NormalTok{g}\SpecialCharTok{+}\FunctionTok{ggtitle}\NormalTok{(}\StringTok{"Histogram of Simulation Mean"}\NormalTok{)}
\NormalTok{g}
\end{Highlighting}
\end{Shaded}

\begin{verbatim}
## `stat_bin()` using `bins = 30`. Pick better value with `binwidth`.
\end{verbatim}

\includegraphics{Statistical-Inference-Project-1_files/figure-latex/unnamed-chunk-7-1.pdf}
The plot indicates that the histogram can be approximated with the
normal distribution \#Comparing Confidence Intervals

\begin{Shaded}
\begin{Highlighting}[]
\NormalTok{sample\_conf }\OtherTok{\textless{}{-}} \FunctionTok{round}\NormalTok{(}\FunctionTok{mean}\NormalTok{(sim\_mean)}\SpecialCharTok{+}\FunctionTok{c}\NormalTok{(}\SpecialCharTok{{-}}\DecValTok{1}\NormalTok{,}\DecValTok{1}\NormalTok{)}\SpecialCharTok{*}\FloatTok{1.96}\SpecialCharTok{*}\FunctionTok{sd}\NormalTok{(sim\_mean)}\SpecialCharTok{/}\FunctionTok{sqrt}\NormalTok{(n),}\DecValTok{3}\NormalTok{)}
\NormalTok{sample\_conf}
\end{Highlighting}
\end{Shaded}

\begin{verbatim}
## [1] 4.729 5.215
\end{verbatim}

\begin{Shaded}
\begin{Highlighting}[]
\NormalTok{theoretical\_conf }\OtherTok{\textless{}{-}} \FunctionTok{round}\NormalTok{(}\FunctionTok{mean}\NormalTok{(theoritical\_mean)}\SpecialCharTok{+}\FunctionTok{c}\NormalTok{(}\SpecialCharTok{{-}}\DecValTok{1}\NormalTok{,}\DecValTok{1}\NormalTok{)}\SpecialCharTok{*}\FloatTok{1.96}\SpecialCharTok{*}\FunctionTok{sqrt}\NormalTok{(theoretical\_var)}\SpecialCharTok{/}\FunctionTok{sqrt}\NormalTok{(n),}\DecValTok{3}\NormalTok{)}
\NormalTok{theoretical\_conf}
\end{Highlighting}
\end{Shaded}

\begin{verbatim}
## [1] 4.755 5.245
\end{verbatim}

The sample 95\% confidence interval {[}4,729,5.215{]} is close to the
theoretical 95\% confidence interval of {[}4.755,5.245{]}

Q-Q Plot for Quantiles

\begin{Shaded}
\begin{Highlighting}[]
\FunctionTok{qqnorm}\NormalTok{(sim\_mean,}\AttributeTok{main =} \StringTok{"Normal Q{-}Q Plot"}\NormalTok{, }\AttributeTok{xlab=}\StringTok{"Theoretical Quantiles"}\NormalTok{,}\AttributeTok{ylab=}\StringTok{"Sample Quantiles"}\NormalTok{)}
\FunctionTok{qqline}\NormalTok{(sim\_mean,}\AttributeTok{col=}\StringTok{"blue"}\NormalTok{)}
\end{Highlighting}
\end{Shaded}

\includegraphics{Statistical-Inference-Project-1_files/figure-latex/unnamed-chunk-10-1.pdf}
Distribution is approximately normal

\end{document}
